\section{Design Pattern Strutturali}
Sono focalizzati sulle dipendenze degli aggetti affichè abbiano certe
caratteristiche

\subsection{Adapter}
\subsubsection{Scopo}
Consente di adattare l'interfaccia di una libreria esterna (Adaptee) ad
un'interfaccia che mi serve (Target) per poterla usare all'interno del mio
sistema
\subsubsection{Struttura}
Due modi per implementarlo:
\begin{itemize}
    \item \textbf{Adapter di classe}: eredito dalla libreria (Adaptee) e implemento l'interfaccia che mi serve (Target)  in una classe ClassAdapter
    \item \textbf{Adapter di oggetto}: creo un oggetto che implementa l'interfaccia (Target) che mi serve e uso Adaptee come attributo
\end{itemize}
\subsubsection{Problematiche}
Piu metodi da implementare, piu complessità

\subsection{Decorator}
\subsubsection{Scopo}
Aggiungere responsabilità ad un oggetto dinamicamente (senza utilizzo di ereditarietà). Permette di aggiungere funzioni e funzionalità ad una funzionalità base. 
\subsubsection{Struttura}
\begin{itemize}
    \item \textbf{Component}: interfaccia che definisce l'oggetto base (Pizza)
    \item \textbf{ConcreteComponent}: implementazione dell'interfaccia Component (BasePizza)
    \item \textbf{Decorator}: classe astratta che implementa l'interfaccia Component e ha un attributo di tipo Component (Topped Pizza)
    \item \textbf{ConcreteDecorator}: implementazione di Decorator (Pizze concrete con aggiunte)
\end{itemize}
\subsubsection{Problematiche}
Tante classi molto simili e piccole possono risultare complesse da testare in isolamento 